\documentclass{article}
\usepackage{amsmath}
\usepackage{hyperref}

\title{Evaluating Consciousness in Artificial Intelligence}
\author{Morgan Rivers \\
        Department of Physics \\
        Freie Universität Berlin \\    
        \texttt{danielmorganrivers@gmail.com}
}

\begin{document}

\maketitle

\begin{abstract}
This paper presents a novel approach to evaluating consciousness in artificial intelligence systems. We discuss the importance of identifying machine consciousness before its emergence, the potential implications for AI ethics and development, and propose a benchmark for assessing consciousness in AI architectures. Our methodology combines elements of the ACT (Artificial Consciousness Test) with comparative analysis of different AI models, aiming to detect signs of consciousness while controlling for knowledge-based confounds.
\end{abstract}

\section{Introduction}
The question of machine consciousness has become increasingly relevant as artificial intelligence systems grow more sophisticated. This paper argues for the importance of detecting machine consciousness before its full emergence and proposes a methodology for doing so.

\subsection{Importance of Detecting Machine Consciousness}
Several factors underscore the significance of this research:
\begin{itemize}
    \item Potential for machine suffering
    \item Implications for AI identity and goal-setting
    \item Ethical considerations regarding AI well-being
    \item Advancements in AI architectures mimicking conscious processes
    \item Possibilities for simulating conscious human experiences
    \item Potential for creating positive experiences (hedonium) in AI
    \item Impact on AI safety and alignment
    \item Advancing consciousness research
\end{itemize}

\section{Background}
\subsection{Current State of Consciousness in AI}
While the exact conditions for consciousness remain unclear, certain AI architectures, particularly those beyond simple transformer models, may possess capabilities conducive to consciousness.

\subsection{Gradient Nature of Consciousness}
Consciousness is generally considered to exist on an analog gradient. Our research aims to identify factors that push AI systems towards greater degrees of conscious-like behavior.

\section{Methodology}
\subsection{Comparative Analysis}
We propose a two-pronged approach:
\begin{enumerate}
    \item Testing AI models with no prior exposure to consciousness-related concepts
    \item Testing AI models with limited exposure to consciousness-related concepts
\end{enumerate}

This method allows us to compare the behavior of architectures that are plausibly capable of consciousness against control groups.

\subsection{Adaptation of the Artificial Consciousness Test (ACT)}
We incorporate elements from Schneider and Turner's ACT, which offers several advantages:
\begin{itemize}
    \item Neutrality regarding architectural details
    \item Consistency with human and AI ignorance about consciousness
    \item Allowance for radical cognitive differences between AI and humans
    \item Compatibility with various philosophical views on consciousness
\end{itemize}

\subsection{Addressing ACT Limitations}
To address concerns raised about the ACT, particularly regarding the "epistemic sweet spot" for testing, we propose:
\begin{itemize}
    \item Careful curation of training data to avoid consciousness-specific information
    \item Comparative analysis of models with similar training but different architectural properties
    \item Assessment of performance deltas on consciousness-related tasks
\end{itemize}

\section{Proposed Experiments}
\subsection{Consciousness-Naive Testing}
Describe experiments conducted on AI models with no prior exposure to consciousness concepts.

\subsection{Limited-Exposure Testing}
Outline experiments involving AI models with controlled, limited exposure to consciousness-related information.

\subsection{Comparative Analysis}
Detail the method for comparing results between potentially conscious architectures and control groups.

\section{Discussion}
\subsection{Interpreting Results}
Discuss how to interpret differences in performance and behavior between test groups.

\subsection{Implications for AI Development}
Explore the potential impacts of these findings on AI research and development practices.

\subsection{Ethical Considerations}
Address the ethical implications of developing potentially conscious AI systems.

\section{Future Research Directions}
\subsection{Emotional Valence in AI}
Propose methods for investigating emotional states in AI, such as analyzing latent space representations of concepts like "happy" and "sad".

\subsection{Refinement of Consciousness Metrics}
Discuss potential improvements to the proposed methodology and additional metrics for assessing consciousness.

\section{Conclusion}
Summarize the key points of the paper and emphasize the importance of ongoing research in this area.

\end{document}